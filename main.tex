\documentclass{article}
\usepackage[utf8]{inputenc}

\title{Differential Forms}
\author{Adrian Liu}
\date{November 2021}

\begin{document}
\setlength{\parindent}{0pt}

\maketitle

\section{Introduction}

In this document I explain differential forms to myself. I do not understand them. The goal is to understand the generalized form of Stoke's Theorem:

\[
\int_{\partial \Omega} \omega = \int_{\Omega} d\omega.
\]

There are only six different symbols in the equation:
\begin{enumerate}
    \item $\int$, an integral 
    \item $\omega$, a differential form 
    \item $\Omega$, an orientable manifold
    \item $\partial$, a boundary operator
    \item $d$, an exterior derivative operator
    \item $=$, an equals sign.
\end{enumerate}

Unfortunately, I only understand the meaning of the last one. So we are going to learn about the remaining symbols one at a time, in the following order: \(\omega, d, \Omega, \partial, \int\). 

\section{Multilinear Algebra}

\end{document}
